\section{Welcome}

\begin{frame}
        {Welcome}

        Hi, I'm Lee Elston, your presenter for this session.

        You can contact me at: \\
        lee@matildasystems.com \\

        I am a course maintainer and instructor for The Linux Foundation
        and maintain: 

        \begin{itemize}

        \item LFS311
        \item LFS430
        \item LFS416
        
        \end{itemize}

\end{frame}

\cprotect\note{

        }



\section{history}

\begin{frame}
   {net-tools}
   \begin{itemize}
      	\item collection of tools for controlling network
	\item in use since kernel version 2.0
	\item most familiar networking commands 
	\item uses procfs (/proc) and ioctl system call 
   \end{itemize}

	\textbf{net-tools} are getting a little old: 
	\url{https://github.com/giftnuss/net-tools} last update 2010. 

\end{frame}

\cprotect\note{


}


\begin{frame}
   {net-tools favorites}

	\center{Some of our favorite \textbf{net-tools} utilities} 

	\begin{longtable}{| m{5em} | m{20em} | }
		\caption{most commonly used net-tools commands} \\
		\hline  
		\textbf{Command}			&
		\textbf{Description} 			\\
		\hline 
		ifconfig				& 
		display and modify interface 		\\ 
		\hline
		netstat					&
		print information about the network 	\\ 
		\hline

	\end{longtable}


\end{frame}

\cprotect\note{
	
}

\begin{frame}
	{The complete list of net-tools:}

	\begin{longtable}{| m{5em} | m{30em} | }
                \caption{net-tools commands} \\
                \hline
                \textbf{Command}                        &
                \textbf{Description}                    \\
                \hline
                ifconfig                                &
                display and modify interface            \\
                \hline
                netstat                                 &
                print information abut the network      \\
                \hline
		arp					&
		print the arp table			\\
		\hline
		ether-wake				&
		send magic "Wake-On-LAN" packet		\\
		\hline
		ipmaddr					&
		add,delete,show multicast addresses 	\\
		\hline 
		iptunnel				&
		add,delete,show a tunnel		\\
		\hline
		mii-daig				&
		network adapter control and monitoring	\\
		\hline
		mii-tool				&
		show,change interface status "Media Independent Interface (MII)	\\
		\hline
		nameif					&
		rename an interface			\\
		\hline
		plipconfig				&
		Parallel Line Internet Protocol (PLIP) 	\\
		\hline
		route					&
		show,add,delete routes			\\
		\hline
		slattach				&
		attach a network interface to a serial line \\
		\hline

        \end{longtable}

\end{frame}

\cprotect\note{

}

\begin{frame}
	{ifconfig} 
	
	\begin{itemize}
	\item
	most used networking command
	\item 
	similar command everywhere
		\begin{itemize} 
		\item Linux
		\item Windows
		\item OSX
		\item Classic Unix and friends  
		\end{itemize}
	\item variations on the name 
		\begin{itemize} 
		\item ipconfig 
		\item ifconfig 
		\end{itemize}
	\item variations on the output 
	\item variations on command line options 

	\end{itemize} 
\end{frame}

\cprotect\note { 
}

\begin{frame} 
	{output of the ifconfig command}
	\begin{raw}
# ifconfig  wlp107s0 
wlp107s0: flags=4163<UP,BROADCAST,RUNNING,MULTICAST>  mtu 1500
        inet 192.168.0.7  netmask 255.255.255.0  broadcast 192.168.0.255
        inet6 fe80::53b1:ef35:6e63:8bab  prefixlen 64  scopeid 0x20<link>
        ether 50:5b:c2:d6:fa:f5  txqueuelen 1000  (Ethernet)
        RX packets 819371  bytes 641332500 (641.3 MB)
        RX errors 0  dropped 0  overruns 0  frame 0
        TX packets 547241  bytes 91562359 (91.5 MB)
        TX errors 0  dropped 0 overruns 0  carrier 0  collisions 0
	\end{raw}
	
\end{frame}

\cprotect\note{ 
}

\begin{frame}
	{netstat}
	
	\begin{itemize}
	\item second most used networking command 
	\item similar command everywhere
		\begin{itemize}
		\item Linux
		\item Windows
		\item OSX
		\item Classic Unix and friends 
		\end{itemize}
	\item variations on the output
	\item variations on command line options 

	\end{itemize}
\end{frame}

\cprotect\note{ 

}

\begin{frame} 
	{output of the netstat command}
\begin{rawsmall}
$ sudo netstat -4tapn 
Active Internet connections (servers and established)
Proto Recv-Q Send-Q Local Address           Foreign Address         State       PID/Program name    
tcp        0      0 192.168.0.7:53          0.0.0.0:*               LISTEN      1076/named          
tcp        0      0 192.168.122.1:53        0.0.0.0:*               LISTEN      1076/named          
tcp        0      0 10.50.0.1:53            0.0.0.0:*               LISTEN      1076/named          
tcp        0      0 127.0.0.1:53            0.0.0.0:*               LISTEN      1076/named          
tcp        0      0 127.0.0.53:53           0.0.0.0:*               LISTEN      832/systemd-resolve 
tcp        0      0 0.0.0.0:22              0.0.0.0:*               LISTEN      1087/sshd           
tcp        0      0 0.0.0.0:25              0.0.0.0:*               LISTEN      2306/master         
tcp        0      0 127.0.0.1:953           0.0.0.0:*               LISTEN      1076/named          
tcp        0      0 0.0.0.0:445             0.0.0.0:*               LISTEN      2129/smbd           
tcp        0      0 0.0.0.0:2049            0.0.0.0:*               LISTEN      -                   
tcp        0      0 0.0.0.0:38657           0.0.0.0:*               LISTEN      1088/rpc.mountd     
tcp        0      0 0.0.0.0:514             0.0.0.0:*               LISTEN      889/rsyslogd        
tcp        0      0 0.0.0.0:111             0.0.0.0:*               LISTEN      1/init              
tcp       32      0 192.168.0.7:35522       50.19.232.147:443       CLOSE_WAIT  2178/opera          
tcp        0      0 192.168.0.7:55912       104.16.40.2:443         TIME_WAIT   -                   
tcp        1      0 192.168.0.7:34522       66.228.47.22:443        CLOSE_WAIT  2178/opera          
tcp        1      0 192.168.0.7:34512       66.228.47.22:443        CLOSE_WAIT  2178/opera          
tcp       32      0 192.168.0.7:60100       107.167.110.216:443     CLOSE_WAIT  2178/opera          
tcp        1      0 192.168.0.7:34510       66.228.47.22:443        CLOSE_WAIT  2178/opera          
tcp        1      0 192.168.0.7:34524       66.228.47.22:443        CLOSE_WAIT  2178/opera          
tcp        0      0 192.168.0.7:46868       172.217.13.202:443      TIME_WAIT   -                   
\end{rawsmall}

\end{frame}

\cprotect\note{ 

}

\section{retirement}


\begin{frame}
	{and now  ......}

	\center{\Huge retirement}

\end{frame}

\cprotect\note{


}	

\begin{frame} 
	{new ideas, new tools} 

	With the rapid change in Linux over the years, some 
	utilities that seem to work are still around, especially the 
	tools we all learned early on in our exposure to all the 
	*ix environments. \\
	So, here we are, nice new tools and we still prefer the old tools. 
			\\ 
	time to look at some alternatives.
	
\end{frame}

\cprotect\note{

	}

\begin{frame}
	{ifconfig vs ip} 
	\begin{rawsmall}
[root@rt ~]# ifconfig ens3
ens3: flags=4163<UP,BROADCAST,RUNNING,MULTICAST>  mtu 1500
        inet 192.168.122.199  netmask 255.255.255.0  broadcast 192.168.122.255
        inet6 fe80::5054:ff:fe0b:1cd6  prefixlen 64  scopeid 0x20<link>
        ether 52:54:00:0b:1c:d6  txqueuelen 1000  (Ethernet)
        RX packets 132952  bytes 181326140 (172.9 MiB)
        RX errors 0  dropped 8487  overruns 0  frame 0
        TX packets 75972  bytes 5166820 (4.9 MiB)
        TX errors 0  dropped 0 overruns 0  carrier 0  collisions 0
	\end{rawsmall}
	\begin{rawsmall}
[root@rt ~]# ip -s  a s  ens3
2: ens3: <BROADCAST,MULTICAST,UP,LOWER_UP> mtu 1500 qdisc fq_codel state UP group default qlen 1000
    link/ether 52:54:00:0b:1c:d6 brd ff:ff:ff:ff:ff:ff
    inet 192.168.122.199/24 brd 192.168.122.255 scope global dynamic ens3
       valid_lft 2751sec preferred_lft 2751sec
    inet6 fe80::5054:ff:fe0b:1cd6/64 scope link 
       valid_lft forever preferred_lft forever
    RX: bytes  packets  errors  dropped overrun mcast   
    181342589  133196   0       8674    0       0       
    TX: bytes  packets  errors  dropped carrier collsns 
    5166820    75972    0       0       0       0   
	\end{rawsmall}

\end{frame}


\cprotect\note{ 

}

\begin{frame}
	{ip flags} 
	One might notice that some additional flags were passed to the \textbf{ip}
	command to make the output look similar to the \textbf{ifconfig} command. 


	This comparison used \textbf{ifconfig} and \textbf{ip -s a s ens3}. \\
	The options for the command \textbf{ip} are: 
		\begin{itemize}
			\item \textbf{-s}, print the link statistics 
			\item \textbf{a}, for address
			\item \textbf{s}, for show 
			\item \textbf{ens3} is the interface
		\end{itemize} 
		Yes, it looks different. 
\end{frame}

\cprotect\note{

}


\begin{frame}
	{ip listing interfaces} 

	The \textbf{most} common use for \textbf{ifconfig} 
	is to check: 
	\begin{itemize}
		\item what is the ip-address 
		\item is the adapter up or down
	\end{itemize}
	How about a shortcut? \\
		\begin{raw}
[root@rt ~]# ip -br a
lo               UNKNOWN        127.0.0.1/8 ::1/128 
ens3             UP             192.168.122.199/24 fe80::5054:ff:fe0b:1cd6/64 
		\end{raw}
\end{frame}

\cprotect\note{

}

\begin{frame}
	{additional command line options for ip}

	This command uses the flag \textbf{-br} for "brief".\\
	The option 
	\textbf{-j} can be used to output the information in "json" format. 
	Adding one more option \textbf{-p} "pretty" makes the json
	output much easier to read. 

	\begin{raw}
[root@rt ~]# ip -br -j -p a s ens3
[ {
        "ifname": "ens3",
        "operstate": "UP",
        "addr_info": [ {
                "local": "192.168.122.199",
                "prefixlen": 24
            },{
                "local": "fe80::5054:ff:fe0b:1cd6",
                "prefixlen": 64
            } ]
    } ]

   	\end{raw}
\end{frame}

\cprotect\note{

}

\begin{frame}
   {netstat vs ss}
	\textbf{netstat} 
	\begin{itemize} 
		\item From the "man" page: \\
			Print network connections, 
			routing tables, interface statistics, 
			masquerade connections, and multicast memberships
		\item Down a bit in that same "man" page: \\
			This program is mostly obsolete.  
			Replacement for netstat is ss.  
	\end{itemize}
	\textbf{ss}
	\begin{itemize}
		\item
 	ss  is used to dump socket statistics. 
		\item
	It allows showing information similar to netstat.
		\item
	It can display more TCP and state information than other tools.
	\end{itemize}

	

\end{frame}

\cprotect\note{

}

\begin{frame}
	{Some netstat replacements}
	\begin{itemize}
	\item Replacement for \textbf{netstat} is \textbf{ss}.  
	\item Replacement for \textbf{netstat -r} is  \textbf{ip route} 
	\item Replacement for \textbf{netstat -i} is \textbf{ip -s} link.
	\item Replacement for \textbf{netstat -g} is \textbf{ip maddr}.

	\end{itemize} 

\end{frame}

\cprotect\note{
}


\begin{frame}
	{ss features} 
	
	The \textbf{ss} program dumps socket statistics. 
	\begin{itemize} 
	\item output is \texttt{similar} to \textbf{netstat}
	\item displays more state information than \textbf{netstat}
	\item displays more \textbf{TCP} information than \textbf{netstat}
	\end{itemize} 

	Many of the options for \textbf{ss} are very 
	similar to \textbf{netstat}.As an example:
	\begin{raw} 
# ss -tuln 
	\end{raw} 
	and
	\begin{raw} 
# nestat -tuln 
	\end{raw} 
	

\end{frame}

\cprotect\note{

}

\begin{frame}
	{compare ss and netstat output} 
	These two commands output almost exactly the same data. 
	\begin{raw}
[root@rt ~]# ss  -ltn4 
State       Recv-Q      Send-Q      Local Address:Port       Peer Address:Port     
LISTEN      0           128               0.0.0.0:5355            0.0.0.0:*        
LISTEN      0           128               0.0.0.0:22              0.0.0.0:*        
LISTEN      0           5               127.0.0.1:631             0.0.0.0:*        
LISTEN      0           100               0.0.0.0:25              0.0.0.0:*        

[root@rt ~]# netstat  -ltn4 
Active Internet connections (only servers)
Proto Recv-Q Send-Q Local Address           Foreign Address         State      
tcp        0      0 0.0.0.0:5355            0.0.0.0:*               LISTEN     
tcp        0      0 0.0.0.0:22              0.0.0.0:*               LISTEN     
tcp        0      0 127.0.0.1:631           0.0.0.0:*               LISTEN     
tcp        0      0 0.0.0.0:25              0.0.0.0:*               LISTEN 
	\end{raw} 
\end{frame}
\cprotect\note{
	}
