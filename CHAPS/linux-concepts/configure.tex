\section{Interface Settings}

\begin{frame}
	{Configure the Interface}
      \begin{itemize}
	      \item \textbf{ip} command
	      \item Edit flat files
	      \item Configuration tools
      \end{itemize}


\end{frame}

\cprotect\note{

}
\begin{frame}
	{The \textbf{ip} Command}
	\begin{raw}
$ ip addr help
Usage: ip address {add|change|replace} IFADDR dev IFNAME [ LIFETIME ]
                                                      [ CONFFLAG-LIST ]
       ip address del IFADDR dev IFNAME [mngtmpaddr]
       ip address {save|flush} [ dev IFNAME ] [ scope SCOPE-ID ]
                            [ to PREFIX ] [ FLAG-LIST ] [ label LABEL ] [up]
       ip address [ show [ dev IFNAME ] [ scope SCOPE-ID ] [ master DEVICE ]
                         [ type TYPE ] [ to PREFIX ] [ FLAG-LIST ]
                         [ label LABEL ] [up] [ vrf NAME ] ]
       ip address {showdump|restore}
IFADDR := PREFIX | ADDR peer PREFIX
          [ broadcast ADDR ] [ anycast ADDR ]
          [ label IFNAME ] [ scope SCOPE-ID ]
SCOPE-ID := [ host | link | global | NUMBER ]
FLAG-LIST := [ FLAG-LIST ] FLAG
	\end{raw}

\end{frame}

\cprotect\note{
   The \textbf{ip} command has many subcommands and can configure
   almost everything about your network interface. The 
   changes made are not persistant. 
}

\begin{frame}
	{\textbf{ip} Command (cont.)}
	\begin{raw}
FLAG  := [ permanent | dynamic | secondary | primary |
           [-]tentative | [-]deprecated | [-]dadfailed | temporary |
           CONFFLAG-LIST ]
CONFFLAG-LIST := [ CONFFLAG-LIST ] CONFFLAG
CONFFLAG  := [ home | nodad | mngtmpaddr | noprefixroute | autojoin ]
LIFETIME := [ valid_lft LFT ] [ preferred_lft LFT ]
LFT := forever | SECONDS
TYPE := { vlan | veth | vcan | vxcan | dummy | ifb | macvlan | macvtap |
          bridge | bond | ipoib | ip6tnl | ipip | sit | vxlan | lowpan |
          gre | gretap | erspan | ip6gre | ip6gretap | ip6erspan | vti |
          nlmon | can | bond_slave | ipvlan | geneve | bridge_slave |
          hsr | macsec
\end{raw}

\end{frame}

\cprotect\note{

}


\begin{frame}
   {Persistant configuration - Debian/Ubuntu}
      \begin{itemize}
         \item /etc/network/interfaces
	 \item /etc/network/interfaces.d/*
      \end{itemize}
	\begin{raw}
auto ens3
iface ens3 inet static
        address 192.168.4.12
        netmask 255.255.255.0
        gateway 192.168.4.1
        dns-nameservers 8.8.8.8
	\end{raw}

\end{frame}

\cprotect\note{

}

\begin{frame}
   {Persistant configuration - Red Hat}
      \begin{itemize}
         \item /etc/sysconfig/network-scripts/ifcfg-eth0
      \end{itemize}
	\begin{raw}
DEVICE=ens4
BOOTPROTO=none
ONBOOT=yes
PREFIX=24
IPADDR=192.168.2.4
	\end{raw}


\end{frame}

\cprotect\note{

}

\begin{frame}
   {Network Manager}
      \begin{itemize}
	 \item nm-connection-editor
	 \item nmtui
         \item nmcli
		 \begin{raw}
$ nmcli connection show WorkNet
connection.id:                          BWOceansidePalms
connection.uuid:                        91733f17-40c3-4c91-b030-cde380715e54
connection.stable-id:                   --
connection.type:                        802-11-wireless
....

$ nmcli con mod WorkNet +ipv4.dns 4.2.2.2
		 \end{raw}
      \end{itemize}


\end{frame}

\cprotect\note{

}


\begin{frame}
   {systemd-networkd}
      \begin{itemize}
         \item Attempt to standardize networking tools
	 \item Lighter weight than NetworkManager
	 \item Uses text files
      \end{itemize}
	\begin{raw}
$ vim /etc/systemd/network/30-wireless.network
[Match]
Name=wlp1s0

[Network]
Address=192.168.8.3/24
Gateway=192.168.8.1
	\end{raw}

\end{frame}

\cprotect\note{

}

