\clearpage\section{Labs}\begin{Lab}

\begin{exe} {Networking Basics}

	\subsection*{Adding A New IP Address}
   We will begin by documenting our network configuration. 
	There are several tools your distribution could use

      \begin{enumerate}
         \item
		 Open a terminal. Use the \textbf{ip} command
		      to view your interfaces and addresses.
		      Save it to a file for later reference.

         \item
		 Determine if \verb:systemd-networkd: is running.

	 \item
		 Determine if \verb:Network Mangager: is 
		      configuring you network.
	      \item
		      View flat file configuration. Your file 
		      name may vary depending on your distribution.

         \item
		 Use the \textbf{ip} command to add a new IP 
		      Address to your primary interface. Your
		      interface name may be different. Typically
		      the interface listed after \verb:lo: in the
		      \textbf{ip} output is your primary. Add the
		      \verb:10.1.2.3: address.

	      \item
		      Use \textbf{ping} to test connection to the
		      new address.
	      \item
		      View the \textbf{ip} details of the interface.
		      Verify the new IP address has been added. Note
		      it is not an alias, but an equal IP as the previous.

	      \item
		      View the route information with \textbf{ip route}.
		      
	      \item
		      Add a new route to \verb:4.2.2.2: via \verb:10.1.2.1:
		      and verify it it is shown in the \textbf{ip route}
		      output.
	      \item
		      Use the \textbf{ifconfig -a} command to view
		      your network settings.  Do you see the new,
		      working IP Address?
	      \item
		      Remove the IP address. After it has been removed, 
		      check to see if the routes have been removed as 
		      well.

      \end{enumerate}

	\subsection*{Changing Parameters}
   In this section we will change the \verb:MTU: and view
	the differences in performance. You can test between 
	two VMs or instances, or between yourself and another
	in class. For ease of use we will test against ourselves.
	Be aware that other network traffic can
	skew results. You may want to test mutliple times.
	Also, some wifi drivers have a lower limit allowed. Your
	interface particulars may be different. 
	\begin{enumerate}
		\item Verify the MTU of your primary interface 
			is \verb:1500:/
		\item Install the Flexible IO, or \textbf{fio}, package.
			We will use this command to generate a 
			known amount of traffic. 
		\item
			Use \textbf{fio} to test network transfer. 
			Copy and paste the summary information to 
			a temporary file or notepad. 
		\item
			Change the \verb:MTU: and run the test again.
			Does the performance change as the \verb:MTU:
			Changes.
	\end{enumerate}

\subsection*{Testing Name Services}
    In this section we will test that we can resolve
    IP addresses, both forward and reverse lookups.
	        \begin{enumerate}
                \item 
			Use the \textbf{host} command to
				view the IP information for
				\verb:Linux.com:.
			\item
				Compare the output with what
				you can see using the \textbf{nslookup} 
				command.
			\item
				Use the \textbf{dig} command to view
				all the DNS information about
				\verb:Linux.com:. 
			\item
				View the name servers for the \verb:Linux.com:
				network.
			\item
				Choose an IP from the output and use 
				\textbf{dig} to resolve the name using the
				IP address.
			\item
				Check your primary interface IP and determine
				if a name is known. Check your route gateway
				as well.
        \end{enumerate}



	        \subsection*{Monitoring Traffic}
		In this section we will use commands to
		view the network traffic.
        \begin{enumerate}
		\item Use \textbf{tcpdump} to view traffic
			on your \verb:lo: interface. Use
			\textbf{cont-C} to stop the command.
			Then view traffic on your primary
			interface.
		\item
			Use a filter to narrow down traffic to
			only port 80, if there is any.
		\item
			Use \textbf{tcpdump} to save traffic
			to a \verb:pcap: file.

		\item
			Install \textbf{Wireshark}. 
		\item
			Use \textbf{sudo} to start
			\textbf{wireshark}.
		\item
			Filter for only SSH traffic. 
		\item 
			Experiment with \textbf{wireshark} as 
			time allows.
        \end{enumerate}

\end{exe}
\end{Lab}
