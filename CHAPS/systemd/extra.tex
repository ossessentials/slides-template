\section{systemd optional features} 

\begin{frame}
	{systemd optional features}
 
\begin{itemize}
                        \item   systemd-delta
                        \item   systemd-timedated
                        \item   systemd-timesyncd
                        \item   systemd-sysctl
                        \item   systemd-networkd
                \end{itemize}
	This is a small sampling of the optional features available with \textbf{systemd}.
	See the official home page at: \\
	\url{https://www.freedesktop.org/wiki/Software/systemd/} \\
	for more information.
\end{frame} 

\cprotect\note{ 

}


\begin{frame}
	{systemd-delta}

	This command/feature \textbf{systemd-delta} is an 
	essential command for administrators, it finds the configuration 
	files that are: 
	\begin{itemize}
		\item overriden
		\item masked
		\item equivalent
		\item redirected
		\item extended
		\item unchanged
	\end{itemize} 

\end{frame}

\cprotect\note{

	}

\begin{frame} 
	{systemd-timedated}

	\begin{itemize} 
		\item  system service that terminates when unneeded
		\item  user interface \textbf{timedatectl}
		\item  change system clock and timezone 
		\item  enable and disable NTP service 
	\end{itemize}

	The \textbf{systemd-timedated} will launch an NTP client when the 
	boolean \textbf{set-ntp true} is set, assuming the previous state 
	was \textbf{false}.\textbf{systemd-timedated} will attempt to launch:
	\begin{itemize} 
		\item \textbf{chronyd.service}
		\item \textbf{ntpd.service} 
		\item \textbf{systemd-timesyncd.service}
	\end{itemize}

	The content of the list and order can be altered by altering the \\
	\textbf{\$SYSTEMD\_TIMEDATED\_NTP\_SERVICES} environment variable. 

\end{frame}

\cprotect\note{

	}


\begin{frame}
	{systemd-timesyncd.service}

	\textbf{systemd-timesyncd.service} is a time synchronization 
	client with the following properties: 
	\begin{itemize}
		\item client only, no server functions
		\item uses \textbf{SNTP} protocol
			\begin{itemize}
				\item "steps" time in to synchronization
				\item uses one time source at a time
			\end{itemize}
		\item configuration files is: \filelink{/etc/systemd/timesyncd.conf}
	\end{itemize} 

\end{frame}

\cprotect\note{
	}


\begin{frame}
	{systemd-sysctl} 

	\textbf{systemd-sysctl} applies kernel parameters 
	early in the boot sequence.  
	\begin{itemize} 
		\item applies the \textbf{.conf} file in \filelink{/etc/sysctl.d} directory
		\item if the original \filelink{/etc/sysctl.conf} is desired 
			create a soft link into the \filelink{/etc/sysctl.d} directory
		\end{itemize}

\end{frame}

\cprotect\note{

	}

\begin{frame} 
	{systemd-networkd}

	\textbf{systemd-networkd} is an optional component that manages networks.
	\begin{itemize}
		\item consistant configuration files across distributions
		\item does not read the historical network configuration files
		\item configuration files in "INI" format 
		\item main configuration files: 
		\filelink{/{lib,run,etc}/systemd/network/} directories see: 
			\textbf{systemd.network(5)} 
		\item low level link settings are independently set
			see \textbf{systemd.link(5)}
		\item non-physical devices are defined in:
			\textbf{systemd.netdev(5)} files
		\end{itemize}

		The default configuration will start a dhcp client on the 
		interfaces at boot time. \\

		Debian and friends have a tool that generates
		either \textbf{NetworkManager} or \textbf{systemd.networkd}
		configuration files from a "yaml" input file.


\end{frame} 

\cprotect\note{

	}

\begin{frame} 
	{test drive systemd-networkd} 
	
	Assuming that \textbf{NetworkManager} is the active configuration 
	tool, testing \textbf{systemd-networkd} only takes a few steps for
	a \textbf{DHCP} example. 
	\begin{itemize} 
		\item create a configuration file
		\item switch off \textbf{netplan} (if applicable)
		\item \textbf{disable} NetworkManager
		\item \textbf{enable} systemd-networkd 
		\item reboot (clears any temporary configs) 
	\end{itemize} 

		Test the new configuration and explore the 
		networking options in \textbf{systemd-networkd} 


\end{frame}

\cprotect\note{

	}
\begin{frame}
	{create a config file} 
	\begin{itemize}
		\item first record the name of the adapter:
	\begin{raw}
$ ip -br address
lo               UNKNOWN    127.0.0.1/8 ::1/128 
enp1s0           UP         192.168.122.42/24 fe80::c2dd:87d:5eb3:9ca5/64 
	\end{raw}
		\item build a config file in the "local administrator"
			directory:
	\begin{raw}
$ cat  /etc/systemd/network/20-dhcp.network 
[Match]
Name=enp*

[Network]
DHCP=yes
	\end{raw} 
		\end{itemize}

\end{frame}

\cprotect\note{

	}

\begin{frame}
	{switch off netplan}

	For this example disable \textbf{netplan} if it is enabled. 

	\begin{itemize}
		\item rename the primary netplan input file directory
		\begin{raw}
# mv /etc/netplan /etc/netplan-off
		\end{raw}

		\end{itemize}

\end{frame}

\cprotect\note{

	}

\begin{frame}
	{switch network management and reboot} 

	Set the automatic start on \textbf{systemd-networkd}
	\begin{itemize}
		\item disable NetworkManager
		\begin{raw}
$ sudo  systemctl disable NetworkManager 
		\end{raw}
		\item enable systemd-networkd
		\begin{raw}
$ sudo systemctl enable systemd-networkd
		\end{raw}
		\item reboot
		\begin{raw}
$ sudo systemctl reboot
		\end{raw}

	\end{itemize}
\end{frame}

	\cprotect\note{

		}




	

	 

