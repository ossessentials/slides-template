\clearpage\section{Labs}\begin{Lab}

\begin{exe} {systemd}

	There is a high probability that your system has \textbf{systemd}
	installed and running. If your system does not then this exercise 
	set will not function.

	This exercise is going to create several configuration files 
	to illustrate how \textbf{systemd} organizes its configuration. 

	Normally administrator added services would be added to \filelink{etc/systemd} 
	directory but in this exercise we are going to create two levels of 
	configuration, the "vendor" and "administration" files.  

	Looking at the man pages to determine the location of the 
	\textbf{systemd} configuration files can be a little complex as \textbf{systemd}
	looks in various places  for its files. Taking a practical 
	approach, we will look at an existing service to determine where 
	your distro has placed the files. 
	
	
      \begin{enumerate}
	      \item An easy way to determine where the \textbf{systemd}
		    configuration files are is to look. \\
		      Locate the configuration files for \textbf{cups} and/or
		      \textbf{sshd}.
		 	\begin{raw}
# systemctl status sshd
			\end{raw}
			The output should be similar to: 
			\begin{raw}
sshd.service - OpenSSH server daemon
   Loaded: loaded (/usr/lib/systemd/system/sshd.service; enabled; vendor preset: disabled)
   Active: active (running) since Sat 2019-08-10 06:27:29 CDT; 2 days ago
     Docs: man:sshd(8)
           man:sshd_config(5)
 Main PID: 687 (sshd)
    Tasks: 1 (limit: 2356)
   Memory: 4.5M
   CGroup: /system.slice/sshd.service
           |_687 /usr/sbin/sshd -D -oCiphers=aes256-gcm@openssh.com,chacha20-poly1305@openssh.com
	
..... some text removed .... 
			
			\end{raw}
	Notice the line beginning with \textbf{Loaded:} as it is 
		      pointing to the initial configuration file 
		      as supplied by the packager. This example
		      shows the directory \filelink{/usr/lib/systemd/system} 
		     is the directory the packager used.  \\  
		   
		      Record the directory name for future use. 
		\item Copy the default configuration to a user home directory. 
			\begin{raw}
# cp /usr/lib/systemd/system/sshd.service /home/student/sshd.service
			\end{raw}

		\item the existing file should look a lot like: 
			\begin{raw}
[root@rt student]# cat sshd.service 
[Unit]
Description=OpenSSH server daemon
Documentation=man:sshd(8) man:sshd_config(5)
After=network.target sshd-keygen.target
Wants=sshd-keygen.target

[Service]
Type=notify
EnvironmentFile=-/etc/crypto-policies/back-ends/opensshserver.config
EnvironmentFile=-/etc/sysconfig/sshd
ExecStart=/usr/sbin/sshd -D $OPTIONS $CRYPTO_POLICY
ExecReload=/bin/kill -HUP $MAINPID
KillMode=process
Restart=on-failure
RestartSec=42s

[Install]
WantedBy=multi-user.target

		\end{raw}
		\item Rename the copy of the sshd.service file to test\_sshd.service 
		\item Alter the contents of test\_sshd.service to match the following: 
			\begin{raw}
# cat /usr/lib/systemd/system/test_sshd.service

[Unit]
Description=test 
After=network.target sshd-keygen.target
Wants=sshd-keygen.target

[Service]
Type=notify
EnvironmentFile=-/etc/crypto-policies/back-ends/opensshserver.config
EnvironmentFile=-/etc/sysconfig/sshd
ExecStart=/usr/sbin/sshd -D $OPTIONS $CRYPTO_POLICY -p 4222
ExecReload=/bin/kill -HUP $MAINPID
KillMode=process

[Install]
WantedBy=multi-user.target

			\end{raw}

		\item The service name is the same as the file name. Copy 
			the customized file back to the default directory. 
		\begin{raw}
# cp test_sshd.service /usr/lib/systemd/system/
		\end{raw}
	\item Tell \textbf{systemd} to re-read its configuration files.
		\begin{raw}
# systemctl daemon-reload
		\end{raw}
	\item The service should be ready to start. Start the new service 
		and request its status.
		\begin{raw}
# systemctl start  test_sshd

# systemctl status test_sshd
test_sshd.service - test
   Loaded: loaded (/usr/lib/systemd/system/test_sshd.service; disabled; vendor preset: disabled)
   Active: active (running) since Mon 2019-08-12 10:50:24 CDT; 2s ago
 Main PID: 5449 (sshd)
    Tasks: 1 (limit: 2356)
   Memory: 1.2M
   CGroup: /system.slice/test_sshd.service
           |_5449 /usr/sbin/sshd -D -oCiphers=aes256-gcm@openssh.com,chacha20-poly1305@openssh.com

Aug 12 10:50:24 rt.example.com systemd[1]: Starting test...
Aug 12 10:50:24 rt.example.com sshd[5449]: Server listening on 0.0.0.0 port 4222.
Aug 12 10:50:24 rt.example.com sshd[5449]: Server listening on :: port 4222.
Aug 12 10:50:24 rt.example.com systemd[1]: Started test.
		\end{raw}
	\item To enable the new service at boot time: 
		\begin{raw}
# systemctl enable test_sshd
		\end{raw} 
	\end{enumerate}
\end{exe} 


\begin{exe}
	{Test a "Drop-In" configuration in the admin directory}

	This exercise will modify the newly created service and will the 
	\textbf{local admin} directory \textbf{/etc/systemd/system}. 
	One of the advantages of using the local admin directory is 
	changes to the service file added by the vendor or 
	packager could be made during update, overwritten local customizations.

	\begin{enumerate} 
		\item
		A Drop-In file can be a single file or a directory 
			of many files, this 
			example uses a directory.
		Create the new directory for the dropin file.
		\begin{raw}
# mkdir /etc/systemd/system/test_sshd.service
		\end{raw}

		\item 
		The Drop-In file will add to the existing service file. Sometimes 
			if a value is already defined it must be cleared before 
			changes can occur, the actual ExecStart command is one. Create 
			dropin file with the following. The port number was changed. 

		\begin{raw}
# cat /etc/systemd/system/test_sshd.service.d/00-first.conf

[Service]
ExecStart=
ExecStart=/usr/sbin/sshd -D $OPTIONS $CRYPTO_POLICY -p4242
		\end{raw}

		\item 
		Inform systemd that the configuration has been changed, then restart the service.
		\begin{raw}
# systemctl daemon-reload 
# systemctl restart test_sshd 
		\end{raw} 

		\item 
		The test\_sshd service should be listening on a new port. 
		\begin{rawfootnotesize} 
[root@rt system]# systemctl status   test_sshd
 test_sshd.service - test
   Loaded: loaded (/usr/lib/systemd/system/test_sshd.service; disabled; vendor preset: disabled)
  Drop-In: /etc/systemd/system/test_sshd.service.d
           |-00-first.conf
   Active: active (running) since Mon 2019-08-12 11:28:32 CDT; 8s ago
 Main PID: 5733 (sshd)
    Tasks: 1 (limit: 2356)
   Memory: 1.0M
   CGroup: /system.slice/test_sshd.service
           |-5733 /usr/sbin/sshd -D -oCiphers=aes256-gcm@openssh.com,chacha20-poly1305@openssh.com
Aug 12 11:28:32 rt.example.com systemd[1]: Stopped test.
Aug 12 11:28:32 rt.example.com systemd[1]: Starting test...
Aug 12 11:28:32 rt.example.com sshd[5733]: Server listening on 0.0.0.0 port 4242.
Aug 12 11:28:32 rt.example.com sshd[5733]: Server listening on :: port 4242.
Aug 12 11:28:32 rt.example.com systemd[1]: Started test.
		\end{rawfootnotesize}



		\end{enumerate}
		\end{exe}


\begin{exe}{Locating the customizations}
	This exercise looks at determining which configuration 
	files are active for a \textbf{systemd unit}. This exercise assumes 
	the successful completion of the previous exercises. 

	\begin{enumerate}
		\item the \textbf{systemctl status test\_vsftpd} command will 
		show the following: 
	\begin{raw}
# systemctl status test_sshd
	\end{raw}

	The output will show:
		\begin{itemize}
		\item service has a vendor default of "enabled" or "disabled"
		\item service has been manually "enabled" or "disabled"
		\item what the name and location of the vendor supplied config file
		\item the Drop-In directory and associated files
		\item limits and cgroup information 
		\item a few lines of the log associated with this service
		\end{itemize}
The output from the test system: 
		\begin{raw}
test_sshd.service - test
   Loaded: loaded (/usr/lib/systemd/system/test_sshd.service; disabled; vendor preset: disabled)
  Drop-In: /etc/systemd/system/test_sshd.service.d
           |-00-first.conf, 60-second.conf
   Active: active (running) since Wed 2019-08-14 08:44:50 CDT; 2s ago
 Main PID: 2629 (sshd)
    Tasks: 1 (limit: 3)
   Memory: 2.2M
   CGroup: /system.slice/test_sshd.service
           |-2629 /usr/sbin/sshd -D -oCiphers=aes256-gcm@openssh.com,chacha20-poly1305@openssh.com,aes256-ct>

Aug 14 08:44:50 rt.example.com systemd[1]: Starting test...
Aug 14 08:44:50 rt.example.com sshd[2629]: Server listening on 0.0.0.0 port 4242.
Aug 14 08:44:50 rt.example.com sshd[2629]: Server listening on :: port 4242.
Aug 14 08:44:50 rt.example.com systemd[1]: Started test.
		\end{raw}

		\item It would be tedious to check every service
			for Drop-In files. \textbf{systemd-delta} 
			is designed to find overridden configuration 
			files. The output is a little wide so 
			redirecting it to a file may be helpful. \\
			
			\begin{raw}
# systemd-delta > /tmp/delta.txt 
			\end{raw}

			display the output:
			\begin{rawscriptsize}
# cat /tmp/delta

[EQUIVALENT] /etc/systemd/system/default.target -> /usr/lib/systemd/system/default.target
[EXTENDED]   /etc/systemd/system/foo.service -> /etc/systemd/system/foo.service.d/00-foo.conf
[EXTENDED]   /usr/lib/systemd/system/systemd-udev-trigger.service -> 
				/usr/lib/systemd/system/systemd-udev-trigger.service.d/systemd-udev-trigger-no-reload.conf
[EXTENDED]   /usr/lib/systemd/system/test_sshd.service -> /etc/systemd/system/test_sshd.service.d/00-first.conf
[EXTENDED]   /usr/lib/systemd/system/test_sshd.service -> /etc/systemd/system/test_sshd.service.d/60-second.conf

5 overridden configuration files found.

			\end{rawscriptsize}
(Note, above edited to fit page) 
	\end{enumerate}



\end{exe}

\begin{exe}{time synchronization with systemd-timedated}

	When synchronization of time is required or when the time client
	client changes the operation must be verified. This exercise will determine
	if time synchronization is active, which client is being used and 
	if necessary 
	set the client to be systemd-timedated

	\begin{enumerate} 
	\item determine which time clients are installed
\begin{raw}
# rpm -qa | grep -e ntp -e chrony -e syncd
yast2-ntp-client-4.1.8-lp151.1.1.noarch
chrony-3.2-lp151.8.6.x86_64
\end{raw}
	This indicates the package \textbf{chrony} is installed. 

	\item 
	Check the current status with \textbf{timedatectl} 
	\begin{raw}
# timedatectl 

      Local time: Wed 2019-08-14 13:00:06 EDT
  Universal time: Wed 2019-08-14 17:00:06 UTC
        RTC time: Wed 2019-08-14 17:45:31
       Time zone: America/Toronto (EDT, -0400)
 Network time on: no
NTP synchronized: no
 RTC in local TZ: no

	\end{raw}
	
	\item

	Disable \textbf{chrony}.
	\begin{raw}
# mv /etc/chrony.conf  /etc/chrony.conf-off 
	\end{raw}

	\item Check if any of the ntp clients are running and stop them.
		\begin{raw}
# systemctl | grep -e ntp -e chrony -e syncd
chronyd.service    loaded active running   NTP client/server

# systemctl | grep -e ntp -e chrony -e syncd
		\end{raw}
		\item enable ntp client with timedatectl
		\begin{raw}
# timedatectl set-ntp on
		\end{raw}

	\item check the status of time synchronization
		\begin{raw}
# timedatectl 
		\end{raw}

		\begin{raw}
      Local time: Wed 2019-08-14 14:02:47 EDT
  Universal time: Wed 2019-08-14 18:02:47 UTC
        RTC time: Wed 2019-08-14 18:02:46
       Time zone: America/Toronto (EDT, -0400)
 Network time on: yes
NTP synchronized: yes
 RTC in local TZ: no
		\end{raw}
			
	\item check which ntp client was started by timedatectl
		\begin{raw}
# systemctl | grep -e ntp -e chrony -e syncd
systemd-timesyncd.service  loaded active running   Network Time Synchronization                                               \end{raw}

	\end{enumerate}

	\end{exe}

	\begin{exe}{switch the network configuration to use systemd-networkd}

	\begin{enumerate}
		\item
		Check the current configuration to verify which network management tools 
		are being used. \\

		Check to see if systemd-networkd is controlling the adapter.
			\begin{raw}
# networkctl 
WARNING: systemd-networkd is not running, output will be incomplete.

IDX LINK             TYPE               OPERATIONAL SETUP     
  1 lo               loopback           n/a         unmanaged 
  2 eth0             ether              n/a         unmanaged 

2 links listed.
			\end{raw}

		Notice the adapters are unmanaged.

	\item
		Check to see if NetworkManager is controlling the adapter.
			\begin{raw}
# nmcli 
eth0: connected to Wired connection 1
        "Red Hat Virtio network device"
        ethernet (virtio_net), 52:54:00:B5:02:FF, hw, mtu 1500
        ip4 default
        inet4 192.168.122.112/24
        route4 0.0.0.0/0
        route4 192.168.122.0/24
        inet6 fe80::3eff:1998:d47:16d8/64
        route6 ff00::/8
        route6 fe80::/64
        route6 fe80::/64

lo: unmanaged
        "lo"
        loopback (unknown), 00:00:00:00:00:00, sw, mtu 65536

DNS configuration:
        servers: 192.168.122.1
        interface: eth0

Use "nmcli device show" to get complete information about known devices and
"nmcli connection show" to get an overview on active connection profiles.
			\end{raw}

	This example is using NetworkManager for the network management.

	\item 
	 Disable NetworkManager and check it is disabled.
		\begin{raw} 
# systemctl disable --now  NetworkManager 

# nmcli 
		\end{raw}
	\item
	Create a new configuration for systemd-networkd.
		\begin{raw}
# cat /etc/systemd/network/20-dhcp.network 
[Match]
Name=eth0

[Network]
DHCP=yes
		\end{raw}

	\item
	Enable systemd-networkd and verify. 
		\begin{raw}
# systemctl enable --now systemd-networkd 
		\end{raw}

		\begin{raw}

# networkctl 
IDX LINK             TYPE               OPERATIONAL SETUP     
  1 lo               loopback           carrier     unmanaged 
  2 eth0             ether              routable    configuring

2 links listed.
		\end{raw}

	Not quite ready yet.
		\begin{raw}
# networkctl 
IDX LINK             TYPE               OPERATIONAL SETUP     
  1 lo               loopback           carrier     unmanaged 
  2 eth0             ether              routable    configured

2 links listed.
		\end{raw}

	\end{enumerate}

\end{exe}
	


\end{Lab}

