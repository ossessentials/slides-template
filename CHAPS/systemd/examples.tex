\section{systemd examples} 

\begin{frame}
	{systemd examples}
 
	This section illustrates some examples of 
	\textbf{systemd} configuration files. 
\begin{itemize}
                        \item   default configuration
                        \item   customize a configuration 

                        \item   systemd-timedated
                        \item   systemd-delta
                        \item   systemd-sysctl
                \end{itemize}
\end{frame} 

\cprotect\note{ 

}

\begin{frame}
	{default configuration}

	Extract the list of files, looking for \textbf{systemd} 
	configuration files. 

\begin{raw}

root@ubuntu:~# dpkg -L vsftpd  | grep systemd 

/lib/systemd
/lib/systemd/system
/lib/systemd/system/vsftpd.service

\end{raw}

	In this package \textbf{vsftpd} the packager has added 
	a \textbf{systemd} configuration file. \\

	Notice the location of the config file. 

\end{frame}

\cprotect\note{


}

\begin{frame}
	{default service config}

	The contents of the default file: 

	\begin{raw} 
root@ubuntu:~# cat /lib/systemd/system/vsftpd.service

[Unit]
Description=vsftpd FTP server
After=network.target

[Service]
Type=simple
ExecStart=/usr/sbin/vsftpd /etc/vsftpd.conf
ExecReload=/bin/kill -HUP $MAINPID
ExecStartPre=-/bin/mkdir -p /var/run/vsftpd/empty

[Install]
WantedBy=multi-user.target
	\end{raw}



\end{frame}

\cprotect\note{


}


\begin{frame}
	{customizing service file} 

	If the packager supplied configuration file 
	is modified it may get overwritten by a package
	update. \\
	There may be additional changes, so a configuration directory 
	will be used. 

	\begin{raw}
root@ubuntu:~# mkdir -p /etc/systemd/system/vsftpd.service.d

root@ubuntu:~# cat /etc/systemd/system/vsftpd.service.d/
		00-vsftpd.conf 
[Service]
ExecStart=
ExecStart=/usr/sbin/vsftpd /etc/vsftpd.conf -oftpd_banner="Whoo Hoo it works"

root@ubuntu:~# systemctl daemon-reload 
root@ubuntu:~# systemctl restart vsftpd 

root@ubuntu:~# ftp localhost
Connected to localhost.
220 Whoo Hoo it works
Name (localhost:student):

	\end{raw}
\end{frame} 

\cprotect\note{

	}

\begin{frame}
	{more complex configuration} 

	The configuration file for vsftpd in 
	\textbf{Fedora} is a little different. 

	\begin{raw}
[root@fedora ~]# rpm  -ql vsftpd | grep systemd
/usr/lib/systemd/system-generators/vsftpd-generator
/usr/lib/systemd/system/vsftpd.service
/usr/lib/systemd/system/vsftpd.target
/usr/lib/systemd/system/vsftpd@.service
	\end{raw}

	Notice the directory is slightly different. \textbf{systemd}
	looks in both directories so it is a a choice.

\end{frame}

\cprotect\note{

	}

\begin{frame}
	{no-frills service file}

	One of the \textbf{service} files supplied with \textbf{vsftpd}
	in \textbf{Fedora} is almost identical to the \textbf{Ubuntu}
	counterpart. 

	\begin{raw}
[root@fedora ~]# cat /usr/lib/systemd/system/vsftpd.service
[Unit]
Description=Vsftpd ftp daemon
After=network.target

[Service]
Type=forking
ExecStart=/usr/sbin/vsftpd /etc/vsftpd/vsftpd.conf

[Install]
WantedBy=multi-user.target
	\end{raw} 

\end{frame}

	\cprotect\note\ {

	
		
	}
		

\begin{frame}
		{some additional features} 

		\begin{raw}
[root@fedora ~]# cat /usr/lib/systemd/system/vsftpd@.service
[Unit]
Description=Vsftpd ftp daemon
After=network.target
PartOf=vsftpd.target

[Service]
Type=forking
ExecStart=/usr/sbin/vsftpd /etc/vsftpd/%i.conf

[Install]
WantedBy=vsftpd.target
		\end{raw}

		Notice:
		\begin{itemize}
			\item
		the \textbf{PartOf} option that connects this 
		service file to the \textbf{vsftp.target}
			\item
		the @ in the name indicates this is a \textbf{template} 
			\item
		the \%i variable that contains the \textbf{instance} name.
		\end{itemize}

\end{frame}

	\cprotect\note{
		

	}

\begin{frame}
	{using a generator}
	
	\begin{itemize}
		\item \textbf{generators} are usually run very early in the 
		initialize process to create or customize configuration 
		files.
	\end{itemize}

	\begin{rawsmall}
[root@fedora ~]# cat /usr/lib/systemd/system-generators/vsftpd-generator
#!/usr/bin/bash

confdir=/etc/vsftpd
unitdir=/usr/lib/systemd/system
targetdir=$1/vsftpd.target.wants

mkdir -p ${targetdir}

for f in $(ls -1 ${confdir}/*.conf | awk -F "." '{print $1}' | awk -F "/" '{print $4}')
do
  echo "Generating systemd units for $f"
  ln -s ${unitdir}/vsftpd\@.service ${targetdir}/vsftpd\@$f.service > /dev/null 2>&1
done

exit 0
	\end{rawsmall}
See \textbf{man 7 systemd.generator} for more information.
	
\end{frame}

\cprotect\note{
	}

\begin{frame}
	{set up generator test}

	\begin{enumerate}
	\item shutdown vsftpd 
	\item copy the config file \filelink{/etc/vsftpd/vsftpd.conf} to 
		\filelink{/etc/vsftpd/blue.conf}
	\item modify the port number in the new file
	\item tell systemd
		\begin{raw}
	# systemctl --system daemon-reload
		\end{raw}
	\item start the vsftpd.target 
	\item confirm the number of active ftp servers
	\end{enumerate}

\end{frame} 

\cprotect\note{

	}





