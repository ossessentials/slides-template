+\section{systemd for Administrators}


\begin{frame}
   {What is systemd}

	\textbf{systemd} is the latest evolution in 
	system startup programs for \textbf{Linux}. \\
		The main objectives are: 
	\begin{itemize}
		\item startup or system manager 
			\begin{itemize}
			\item runs as "init" or "pid 1"
			\item launched by the kernel during initialization
			\item starts the rest of the system
			\end{itemize}
		\item session or service manager 
			\begin{itemize}
			\item start, stop and status of background processes
			\item SysV init.d script compatible
			\end{itemize}
	\end{itemize}

\end{frame}

\cprotect\note{

}

\begin{frame}
	{service control commands}
	\textbf{systemd} has a control program \textbf{systemctl} to manage
	the background processes and services it starts. Some of the options are: 
	\begin{itemize}
	\item \textbf{systemctl}  outputs a list of all "units" and their state
	\item \textbf{systemctl -t services} list the "service units" and their state
	\item \textbf{systemctl status cups} show detailed status for "cups.service" \\
		(the .service is default other types mus be specified)
	\item \textbf{systemctl start cups} start the "cups.service" now 
	\item \textbf{systemctl stop cups} stop the "cups.service" now
	\item \textbf{systemctl enable cups} start the "cups.service" at boot time
	\item \textbf{systemctl disable cups} do not auto start this service at boot time
	\end{itemize}
	Please consult \textbf{man systemctl} for additional details. 
\end{frame}

\cprotect\note{

}

\begin{frame}
	{\textbf{systemd} configuration directories}
	systemd has a three tiered configuration, if one of the 
	tiers is not present the next highest priority files is 
	used. 
  	\begin{itemize} 
        \item Configuration files/directories, priority lowest to highest.
                \begin{itemize}
                \item vendor/packager
                \item dynamic
                \item administrator
                \end{itemize}
        \end{itemize}

\end{frame}

\cprotect\note{

}

\begin{frame}
	{The typical directory locations} 
	\begin{itemize}
	\item the vendor supplied configuration file is at: \\
		\filelink{/lib/systemd/system/<service-name>.service}  \\
			or \\
		\filelink{/usr/lib/systemd/system/<service.name>.service} \\
			or \\
		\filelink{/usr/lib/systemd/system/<service-name>.d/*.conf}

	\item temporary files created at runtime:  \\
		\filelink{/run/systemd/} 

	\item local administrator configuration files:  \\
		\filelink{/etc/systemd/system}	
	\end{itemize}
\end{frame}

\cprotect\note{ 

} 

\begin{frame} 
	{configuration file types}

	\textbf{systemd} has several types of configuration elements, called \textbf{units}.
	   \begin{longtable}{| m{5em} | m{30em} | }
                \caption{some of the unit types} \\
                \hline
		   \textbf{Type}                        &
                \textbf{Description}                    \\
                \hline
                service                                &
               which start and control daemons and the processes they consist of  \\
                \hline
		   socket &
                local IPC  or network socket      \\
                \hline
		target &
		group of units or boot up points \\
		\hline
		 device &
		 exposes kernel devices to systemd, allows device based activation \\
		   \hline
		   mount &
		   control the mount points \\
		   \hline
		   automount &
		   provides on demand mounting of filesystems \\
		   \hline
		   timer  &
		   trigger activation of other units based on timer\\
		   \hline
		   swap &
		   mount swap files \\
		   \hline
		   path & 
		   activate other units if filesystem objects change \\
		   \hline 
		   slice &
		   group units for easier management \\
		   \hline
		   scope &
		   like service units but manage foreign processes as well \\
		   \hline
		   

        \end{longtable}

\end{frame}

\cprotect\note{

}

\begin{frame}
	{"Unit" documentation}
 \begin{longtable}{| m{5em} | m{30em} | }
                \caption{unit type man pages} \\
                \hline
                   \textbf{Type}                        &
		\textbf{man page}                    \\
                \hline
                service                                &
	       systemd.service(5) \\
                \hline
                   socket &
		systemd.socket(5), daemon(7)     \\
                \hline
                target &
		systemd.target(5) \\
                \hline
                 device &
		 systemd.device(5) \\
                   \hline
                   mount &
		   systemd.mount(5) \\
                   \hline
                   automount &
		   systemd.automount(5) \\
                   \hline
                   timer  &
		   systemd.timer(5) \\
                   \hline
                   swap &
		   systemd.swap(5) \\
                   \hline
                   path &
		   systemd.path(5) \\
                   \hline
                   slice &
                   systemd.slice(5) \\
                   \hline
                   scope &
		   systemd.scope(5) \\
                   \hline
		   "see also" &
		   systemd.special(7), systemd.units(5), systemd-system.conf(5) \\ 
		   \hline


        \end{longtable}

\end{frame}

\cprotect\note{

}

