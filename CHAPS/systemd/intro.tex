\section{Introduction to systemd for Administrators}


\begin{frame}
	{Welcome}

	Hi, I'm Lee Elston, your presenter for this session.

	You can contact me at: \\
	lee@matildasystems.com \\

	I am a course maintainer and instructor for The Linux Foundation
	and maintain: 

	\begin{itemize}

	\item LFS311
	\item LFS430
	\item LFS416
	
	\end{itemize}

\end{frame}

\cprotect\note{

	}


\begin{frame}
   {Introduction}

	This presentation will focus on the following \texttt{administration}
	aspects of \textbf{systemd} including the following: 
   \begin{itemize}
	\item What is \textbf{systemd}
	\item Getting your system services started
	\item Configuration files/directories 
	\item builtin and optional services 
    \end{itemize}

\end{frame}

\cprotect\note{


}

\begin{frame} {history}

        To start a Unix/Linux system we need an initialization program
        to start the system environment. We also would like an easy
        method to launch services both system and user
        related. Several methods were used:

        \begin{itemize}
        \item An initialization program and one
                big script running on the default shell
        \item An initialization program and a bunch of little scripts
        \item An initialization program and the old bunch of scripts
                        with the beginnings of
                        dependency processing.
        \item An initialization program no scripts, just configuration files
                with dependencies and sequencing available.
        \end{itemize}


\end{frame}

\cprotect\note{


}


